\newpage % Zaleca się otwieranie rozdziału od nowej strony.
\section{State~of~Art}
\label{sc:state}

\subsection{GPU}
Procesory graficzne są dedykowane do wykonywania wielu równoległych obliczeń.
Dzięki temu, są bardzo wydajne w zadaniach które można łatwo zrónoleglić.
Wiele algorytmów do kryptoanalizy pozwala na przetwarzanie równoległe, 
w szczególności algorytm \textbf{rho-Pollarda}.
\subsubsection*{Solving Discrete Logarithms in Smooth-Order Groups with CUDA}
W roku 2012 na karcie graficznej NVIDIA Tesla M2050 osiągnięto wydajność na poziomie
51.9 miliona operacji mnożenia modularnego 768-bit na sekundę.
Implementacja opierała się głównie na języku C z CUDA framework wraz z jednostkowymi segmentami
w języku PTX który jest zbiorem instrukcji dla CUDA GPU.
Praca ma dla mnie szczególną na tym etapie, ponieważ razem z pracą udostępniono kod
implementacji na prawach open-source, dodatkowo opisuje
ograniczenia i założenia jakie należy uwzględnić przy implementacji
algorytmu rho-Pollarda na GPU\cite{henry-goldberg-cuda}.

\subsubsection*{ECC2K-130 on NVIDIA GPUs}
Artykuł opisuje implementację algorytmu rho-Pollarda na karcie graficznej NVIDIA 
GTX 295.
Autorzy wybrali krzywą Koblitza ECC2K-130.
Opisano decyzje związane z wyborem bazy ( w tym przypadku wybrano bazę normalną).
Przedstawiono również szczegóły związane z zarządzaniem pamięcią oraz problem związany
z DRAM'em karty (przy pełnej utylizacji GPU w pamięci brakowało miejsca na input)
Wynik: Średnio obliczenie ECDLP na tej krzywej zajełoby 2 lata przy 534 kartach.

\subsection{FPGA}
\subsubsection*{Solving Discrete Logarithms in Smooth-Order Groups with CUDA}
W 2014 opublikowano pracę przedstawiającą implementację FPGA
na platformie Virtex-6.
dedykowaną do rozwiązania logarytmu dyskretnego na 113-bitowej krzywej Koblitza.
Opisano zastosowane zabiegi poprawiające optymalizację, oraz design
poszczególnych moduł
Na przykład w celu lepszej optymalizacji, wykorzystano bazę normalną
$F_{2^m}$ w jednym z modułów do liczenia automorfizmu punktów.
Wynik po ekstrapolacji to 28 dni na rozwiązanie logarytmu na krzywej Koblitza 113 bit.

\subsection{CPU}
CPU nie są najwydajniejszą architekturą do wykonywania równoległych obliczeń.
Zazwyczaj charakteryzują się znacznie wydajniejszymi jednostkami obliczeniowymi (rdzeniami)
niż na przykład GPGPU, ale jest ich również znacznie mnniej niż w GPGPU.
CPUs są najlepiej przystosowane do przetwarzania potokowego.
\subsubsection*{A Review on solving ECDLP over Large Finite Field using Parallel
	Pollard's Rho (p) Method}
Praca przedstawia wyniki czasowe przy obliczaniu ECDLP na
ciele skończonym rzędu p do 85-bitów.
Zastosowano do tego cluster CPU o 256 rdzeniach octa-core.
Artykuł również jest interesujący ponieważ zwięźle opsuje background matematyczny
oraz przejrzyście przedstawia wersję równoległą
algorytmu rho Pollarda\cite{rewiev-elliptic-cpu}.
Wynik to 52 godziny dla krzywej na ciele rozmiaru p = 85-bitów.


