\newpage
\section{Wprowadzenie}

Wraz z rosnącą potrzebą zapewnienia poufności i integralności danych w systemach
informatycznych, kryptografia stała się jednym z filarów współczesnych rozwiązań
bezpieczeństwa. Kryptografia oparta na krzywych eliptycznych \textbf{ECC} ( \textit{Elliptic Curve Cryptography}) jest
obecnie jednym z najważniejszych standardów w tej dziedzinie, oferując wysoki
poziom bezpieczeństwa przy znacznie mniejszych rozmiarach kluczy w porównaniu
do tradycyjnych algorytmów, takich jak RSA \cite[]{Ao2016,Barker2016}.
Z tego powodu ECC znajduje szerokie zastosowanie w systemach o ograniczonych zasobach,
urządzeniach IoT ( \textit{Internet of Things}) czy urządzeniach mobilnych \cite[]{Thakur2022,Hammi2020}.

Bezpieczeństwo ECC opiera się na trudności rozwiązania problemu logarytmu
dyskretnego na krzywych eliptycznych. Jest to problem matematyczny, który w
praktyce nie ma efektywnego rozwiązania w rozsądnym czasie przy użyciu
współczesnych metod obliczeniowych. Do tej pory nie opracowano algorytmu
zdolnego rozwiązać ten problem w czasie podwykładniczym na komputerach klasycznych, dlatego
kryptoanaliza ECC wymaga dużych zasobów obliczeniowych \cite{Menezes2001}.
Z tego powodu, często wykorzystywane w tej roli są układy GPU \cite[]{Boss2015,Panetta2017,Bernstein2012},
FPGA \cite[]{Wenger2014,Mane2011,FPGA2008,Majkowski2008}
czy nawet konsole dla graczy \cite{Bos2010}.

Gwałtowny rozwój kart graficznych \textbf{GPU} ( \textit{Graphic Processing Unit})
oraz technologii \textbf{CUDA} ( \textit{Compute Unified Device Architecture}) znacznie zwiększył możliwości
przeprowadzania takich analiz. Choć rozwój ten był w dużej mierze napędzany
zapotrzebowaniem związanym z uczeniem maszynowym, jego zastosowanie w
innych dziedzinach, takich jak kryptoanaliza jest równie istotny.

Karty graficzne, dzięki architekturze zoptymalizowanej do równoległych obliczeń,
stały się narzędziem o kluczowym znaczeniu w dziedzinach wymagających intensywnego
przetwarzania danych. Platforma CUDA, opracowana przez firmę NVIDIA, umożliwia
wykorzystanie mocy obliczeniowej GPU do realizacji zadań takich jak
implementacja algorytmów kryptograficznych oraz analiza ich odporności na ataki.
CUDA pozwala na masywnie równoległe przetwarzanie danych, co ma kluczowe znaczenie
w implementacji algorytmu Rho Pollard'a, który w wersji równoległej, pozwala na efektywne
wykorzystanie możliwości GPU.

Chociaż GPU nie oferują wydajności i efektywności energetycznej 
porównywalnej z dedykowanymi układami, takimi jak FPGA czy ASIC, ich dostępność 
i stosunkowo niski koszt czyni je atrakcyjnym wyborem do obliczeń kryptograficznych. 
W obliczu rosnącej mocy obliczeniowej GPU oraz ich szerokiej dostępności, ciągła 
analiza odporności algorytmów kryptograficznych na ataki staje się jeszcze 
ważniejszym elementem ich rozwoju.

\subsection{Cel pracy}
Celem tej pracy jest realizacja systemu korzystającego z koprocesora GPU,
w celu przyśpieszenia obliczeń przy
rozwiązywaniu problemu logarytmu dyskretnego.
Implementacja została zoptymalizowana i dostosowana do obliczenia logarytmu
dyskretnego na krzywej eliptycznej ECCp-79, zaproponowanej w wyzwaniu \textit{The Certicom ECC Challenge} \cite{certicom-challange}.
\par
Głównymi elementami stworzonego systemu
jest program klienta wykonującego część algorytmu Rho Pollard'a na karcie graficznej Nvidia
z wykorzystaniem technologii CUDA i języka programowania CUDA C++,
oraz program serwera działający na CPU, odpowiedzialny za zarządzanie klientami i zbieranie wyników.
 Wyniki implementacji zostały porównane z innymi pracami
których celem była kryptoanaliza ECC z wykorzystaniem akceleracji sprzętowej.