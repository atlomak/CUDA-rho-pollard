\newpage
\section{Podsumowanie oraz dalsze usprawnienia}

Realizacja pracy pozwoliła mi na zgłębienie zagadnień związanych ze
współczesną kryptografią oraz jej praktyczną implementację. Dodatkowo,
poznałem techniki akceleracji obliczeń z wykorzystaniem GPU. Efekty mojej pracy
uważam za satysfakcjonujące, ponieważ udało się zrealizować wszystkie
założone cele. Obliczenie poprawnego rozwiązania dla wyzwania Certicom,
osiągając wydajność porównywalną z podobnymi pracami, uważam za spory sukces,
biorąc pod uwagę, że była to moja pierwsza styczność z technologią CUDA oraz
kryptografią opartą o krzywe eliptyczne. W pracy pozostaje jednak miejsce na
potencjalne optymalizacje i usprawnienia.

Wykorzystanie lżejszych bibliotek niż pakiet SageMath pozwoliłoby na
łatwiejsze uruchomienie rozwiązania na innych platformach, bez potrzeby
czasochłonnej kompilacji całego pakietu obliczeniowego, którego znaczna
część nie jest wykorzystywana.
Zastosowanie redukcji Barrett'a do operacji dzielenia modulo $p$ pozwoliłoby
jeszcze bardziej zwiększyć wydajność obliczeń na ciele. Ponieważ wszystkie
obliczenia są wykonywane na tych samych parametrach ciała, przeliczenie
specjalnej wartości wymaganej przez tę technikę odbyłoby się tylko raz na
początku obliczeń.

Kolejnym miejscem na potencjalne usprawnienia jest optymalizacja wykorzystania 
rejestrów w kernel'u CUDA. Obecna implementacja ma niewielką liczbę rejestrów,
które musiały być przerzucone przez kompilator do wolniejszej pamięci
globalnej. Prawdopodobnie uważna optymalizacja oraz dokładniejsza analiza z
pomocą profiler'a \textit{NVIDIA Nsight Systems} pozwoliłaby zredukować ilość
"rozlanych" rejestrów, ograniczając narzut związany z niższą
przepustowością pamięci globalnej, oraz zmniejszyć liczbę rozgałęzień w kodzie.

Praca pokazała, że GPU umożliwia osiągnięcie wysokiej wydajności w obliczeniach
kryptograficznych, jednak pomimo rosnącej mocy obliczeniowej oraz wysokiej dostępności
GPU, rozmiar kluczy stosowanych we współczesnej kryptografii nadal zapewnia
odpowiedni poziom bezpieczeństwa. W efekcie złamanie kryptosystemów opartych na
ECC w rozsądnym czasie pozostaje praktycznie niemożliwe.