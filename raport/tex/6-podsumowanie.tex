\newpage  
\section{Podsumowanie oraz dalsze usprawnienia}  
Celem tej pracy było zaimplementowanie systemu,
który z wykorzystaniem GPU jest w stanie rozwiązać ECDLP na ECCp79 w jak najkrótszym czasie.
\par
Aby osiągnąć ten cel, niezbędne było zapoznanie się z wieloma publikacjami, które
przedstawiają współczesne sposoby optymalizacji obliczeń kryptograficznych oraz
najwydajniejsze techniki projektowania systemów do kryptoanalizy krzywych eliptycznych.
Pozwoliło to na zastosowanie bardziej optymalnego algorytmu Rho Pollard'a, w wersji Addition walk,
który znacznie lepiej działa w przypadku architektury SIMD jaką jest GPU.
Wykorzystanie optymalizacji, takich jak Montgomery Trick, również pozwoliło
jeszcze bardziej zwiększyć wydajność obliczeń.

Opracowana koncepcja oraz opis implementacji, szczegółowo przedstawiają sposób
działania zarówno głównego programu do obliczeń kryptograficznych, jak i serwera
wraz z całą architekturą niezbędną do uruchomienia całego systemu.
Wszelkie napotkane trudności oraz decyzje projektowe, zostały dokładnie opisane,
co pozwala na odtworzenie podobnego systemu oraz wyjaśnia niektóre zabiegi implementacje.
Wszystkie moduły składające się na system, zostały dokładnie przetestowane oraz zweryfikowane przy
pomocy szerokiej gamy testów. Dodatkowo działanie zostało zweryfikowane
poprzez prawidłowe rozwiązanie założonego problemu ECDLP.

Praca spełniła wszystkie założone cele, osiągając przy tym satysfakcjonujące wyniki.
Poprawne rozwiązanie wyzwania Certicom, przy jednoczesnym osiągnięciu wydajności porównywalnej z  
podobnymi pracami, stanowi istotny sukces.
Realizacja pracy pozwoliła na zgłębienie zagadnień związanych ze współczesną  
kryptografią, jej praktyczną implementacją oraz technikami  
akceleracji obliczeń z wykorzystaniem GPU.  

Wyniki zademonstrowane w pracy pokazały, że GPU umożliwia osiągnięcie
wysokiej wydajności w obliczeniach kryptograficznych.
Pomimo rosnącej mocy obliczeniowej,
rozmiar kluczy stosowanych we współczesnej kryptografii nadal zapewnia
odpowiedni poziom bezpieczeństwa. W efekcie złamanie kryptosystemów opartych na
ECC w rozsądnym czasie, nawet wykorzystując wiele układów GPU, pozostaje praktycznie niemożliwe.

\subsection{Krytyczna ocena oraz dalsze usprawnienia}
Pomimo uzyskania zadowalających wyników,
w pracy nadal pozostaje przestrzeń na dalsze optymalizacje i usprawnienia.
\par
Zastosowanie lżejszych bibliotek niż pakiet SageMath mogłoby uprościć  
uruchamianie rozwiązania na innych platformach, eliminując konieczność  
czasochłonnej kompilacji rozbudowanego pakietu obliczeniowego, którego znaczna  
część pozostaje niewykorzystana.
\par
Wprowadzenie redukcji Barrett'a do operacji  
dzielenia modulo $p$ w programie GPU, mogłoby dodatkowo zwiększyć wydajność obliczeń w ciele.  
Ponieważ wszystkie operacje wykonywane są na tych samych parametrach ciała,
przeliczenie specjalnej wartości wymaganej przez tę technikę byłoby konieczne 
jedynie na początku obliczeń, znacznie upraszczając implementację.
Dodatkowo, odpowiednie zastosowanie Negation Map, pozwoliłoby znacząco skrócić czas obliczeń.
Jednak zaimplementowanie tej techniki, nie jest trywialne w architekturze SIMD \cite{Negation}.
\par
Kolejnym miejscem na potencjalne usprawnienia jest optymalizacja wykorzystania 
rejestrów w kernel'u CUDA. Obecna implementacja ma niewielką liczbę rejestrów,
które musiały być przeniesione przez kompilator do wolniejszej pamięci
globalnej. Prawdopodobnie uważna optymalizacja oraz dokładniejsza analiza z
pomocą profiler'a \textit{NVIDIA Nsight Systems} pozwoliłaby zredukować ilość
\textit{spilled registers}, ograniczając narzut związany z niższą
przepustowością pamięci globalnej, oraz zmniejszyć liczbę rozgałęzień w kodzie.
\par
Mniej istotnym z punktu widzenia wydajności (obliczeniowej),
ale możliwym do poprawy miejscem, jest organizacja kodu.
Z racji \textit{jednorazowego} charakteru tej implementacji,
czystość i struktura kodu nie była priorytetem podczas jej rozwoju.
Osoba bardziej doświadczona w środowisku programowania C++ oraz CUDA, prawdopodobnie
znalazłaby kilka miejsc, w których zastosowanie innego podejścia dałoby lepsze rezultaty
oraz ułatwiłoby potencjalną modyfikację.