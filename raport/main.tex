%%%%%%%%%%%%%%%%%%%%%%%%%%%%%%%%%%%%%%%%%%%%%%%%%%%%%%%
%% Engineer & Master Thesis, LaTeX Template          %%
%% Copyleft by Piotr Woźniak & Artur M. Brodzki      %%
%% Faculty of Electronics and Information Technology %%
%% Warsaw University of Technology, Warsaw, 2019     %%
%%%%%%%%%%%%%%%%%%%%%%%%%%%%%%%%%%%%%%%%%%%%%%%%%%%%%%%

\documentclass[
    left=2.5cm,         % Sadly, generic margin parameter
    right=2.5cm,        % doesnt't work, as it is
    top=2.5cm,          % superseded by more specific
    bottom=3cm,         % left...bottom parameters.
    bindingoffset=6mm,  % Optional binding offset.
    nohyphenation=true % You may turn off hyphenation, if don't like. =false
]{eiti/eiti-thesis} % bazuje na clasie mwart

\nocite{*}

\usepackage[
    backend=bibtex,
]{biblatex}
\usepackage{csquotes}
\usepackage{hyperref}
\usepackage{algpseudocode}
\usepackage{algorithm}

\langpol % Dla języka angielskiego mamy \langeng
\graphicspath{{img/}}             % Katalog z obrazkami.
\addbibresource{bibliografia.bib} % Plik .bib z bibliografią

\makeatletter
\xpatchcmd\lst@MakeCaption{\protect\numberline{\thelstlisting}\lst@@caption}{\protect\numberline{\thelstlisting.}\lst@@caption}{}{}
%\makeatother
%\makeatletter
%\xpatchcmd{\LT@c@ption}{\protect\numberline{\thetable}}{\protect\numberline.{. \thetable . }}{}{}
\makeatother

\begin{document}

%--------------------------------------
% Strona tytułowa
%--------------------------------------
\RaportThesis
\instytut{Cyberbezpieczeństwa}
\kierunek{Telekomunikacja}
\specjalnosc{Techniki Teleinformatyczne}
\title{
    Akceleracja sprzętowa kryptoanalizy algorytmów kryptograficznych
}
\engtitle{ % Tytuł po angielsku do angielskiego streszczenia
    Hardware acceleration of cryptoanalysis of cryptograhpic algorithms
}
\author{Andrzej Tłomak}
\album{311450}
\promotor{dr. hab. inż. Mariusz Rawski}
\date{\the\year}
\maketitle

%--------------------------------------
% Streszczenie po polsku
%--------------------------------------
\streszczenie
Cel pracy
\slowakluczowe Krzywe eliptyczne, Kryptografia, Kryptoanaliza, CUDA, Algorytm~rho~Pollard'a

\newpage

%--------------------------------------
% Streszczenie po angielsku
%--------------------------------------
\abstract
The objective of this phase included a review of the literature describing the current State-of-Art \
in cryptoanalysis of systems \
based on Elliptic curves in Finite Fields. It involved getting deeper knowledge of theory
and mathematical fundation of Elliptic curves as well as setting up an environment
to develop implementation utillizing CUDA technology.
\keywords Elliptic curves, Cryptography, Cryptoanalysis, CUDA, FPGA, rho~Pollard~algorithm
\newpage

%--------------------------------------
% Oświadczenie o autorstwie
%--------------------------------------
\makeauthorship
\blankpage

%--------------------------------------
% Spis treści
%--------------------------------------
%\thispagestyle{empty}
\tableofcontents
%\blankpage

%--------------------------------------
% Rozdziały
%--------------------------------------
\newpage
\section{Wprowadzenie}
Celem tej pracy była implementacja akecleracji sprzętowej algorytmu do kryptoanalizy kryptosystemów opartych
o problem logarytmu dyskretnego. Jednym z prostszych sposobów akceleracji algorytmów które
pozwalają na ich zrównoleglenie, jest wykorzystanie procesorów graficznych GPGPU.
Praca ta skupia się na wykorzystaniu framework'u Nvidia CUDA wraz z kartą graficzną Nvidia GTX 2070 Super
do przyśpieszenia kryptoanalizy krzywej ECCp-79 z listy Certicom.

\newpage % Zaleca się otwieranie rozdziału od nowej strony.
\section{Wstęp teoretyczny}


\subsection{Procesor graficzny}
DO UZUPEŁNIENIA
GŁÓWNIE SŁOWNICTWO oraz architektura SIMD
SM - streaming multiprocesor
BLOCK - grupa wątków, najbardziej granularny sposób rozpoczynania obliczeń
wątek - pojedyncza logiczna jednostka obliczeniowa, wykonuje instrukcje w grupach po 32
(rozgałęzienia powodują sekwencyjność w tej grupie 32) 
good practice: brak rozgałęzień w kodzie, jedna instrukcja, wiele danych

Hierarchia pamięci

\subsection{Krzywe eliptyczne}
Zakładając, że ciało $\mathbb{K}$ ma charakterystykę różną od 2 i 3,
oraz że stałe $a, b \in \mathbb{K}$ spełniają warunek:
\[4a^3 + 27b^2 \neq 0\]
nieosobliwą krzywą eliptyczną nad ciałem $\mathbb{K}$ definiuje się jako zbiór punktów $(x,y) \in \mathbb{K} \times \mathbb{K}$,
spełniających równanie:
\[y^2 = x^3 + ax + b\]
wraz ze specjalnym punktem w nieskończoności $\mathcal{O}$, który pełni rolę elementu neutralnego
w działaniach grupowych
\cite{Stinson2021}.
\begin{figure}[H]
    \centering \includegraphics[width=0.8\linewidth]{sage/krzywa_-4_2.png}
    \caption{Krzywa eliptyczna $y^2=x^3-4+2$ nad ciałem liczb rzeczywistych}
\end{figure}

Krzywe eliptyczne zdefiniowane na liczbach rzeczywistych nie są kluczowe w
systemach kryptograficznych \cite*{Stinson2021}, ale takie ustawienia
pozwalają na prostsze przedstawienie niektórych zagadnień
np. dodawnie punktów na krzywej.


\subsubsection{Dodawanie punktów na krzywej eliptycznej}
Odpowiednie zdefiniowanie operacji dodawania punktów na krzywej eliptycznej
pozwala otrzymać grupę abelowa, złożoną z punktów krzywej oraz punktu w nieskończoności jako
elementu neutralnego.
\newline
\indent
Geometrycznie, dodawanie punktów na krzywej eliptycznej nad ciałem liczb rzeczywistych można przedstawić
jako połączenie dwóch punktów $P$ i $Q$ prostopadłą linią, która przecina krzywą w trzecim
punkcie, $R'$. Następnie, wynikowy punkt $R$, będący sumą $P+QP+Q$, znajdujemy przez
odbicie punktu $R'$ względem osi $x$. W przypadku podwojenia punktu, czyli dodawania
punktu P do siebie samego, rysujemy styczną do krzywej w punkcie $P$, która przecina
krzywą w nowym punkcie. Odbicie tego punktu względem osi $x$ daje nam wynik $2P$ \cite{Chrzeszczyk2010}\cite{Stinson2021}.
\begin{figure}[!h]
    \centering \includegraphics[width=0.8\linewidth]{sage/elliptic_rational_point_addition.png}
    \caption{P + Q na krzywej eliptycznej $y^2+y=x^3-x^2+2x$}
\end{figure}
\par
Definując dodawanie punktów na krzywej eliptycznej w sposób algebraiczny otrzymujemy następujące wzory:
\begin{enumerate}
    \item Przypadek, gdy \( P \neq Q \):
          \begin{align}
              \lambda & = \frac{y_2 - y_1}{x_2 - x_1}, \\
              x_3     & = \lambda^2 - x_1 - x_2,       \\
              y_3     & = \lambda(x_1 - x_3) - y_1
          \end{align}
    \item Przypadek, gdy \( P = Q \):
          \begin{align*}
              \lambda & = \frac{3x_1^2 + a}{2y_1}, \\
              x_3     & = \lambda^2 - 2x_1,        \\
              y_3     & = \lambda(x_1 - x_3) - y_1
          \end{align*}
    \item Szczególny przypadek, gdy \( P = -Q \):
          \begin{align*}
              P + (-P) = \mathcal{O}
          \end{align*}
\end{enumerate}
Dodatkowo odwrotność punktu na krzywej $P$ definujemy jako $-P = (x, -y)$ \cite{Stinson2021}.


\subsubsection{Krzywe eliptyczne na ciele skończonym}
Krzywe eliptyczne zdefiniowane na ciele skończonym $F_p$ oraz $F_{p^n}$ mają kluczowe znaczenie w kryptografii.
W swojej pracy skupiłem się wyłącznie na krzywych zdefiniowanych na ciele skończonym $F_p$ gdzie $p$ jest liczbą pierwszą.
\par
Wykres krzywej eliptycznej nad ciałem $F_p$ nie przypomina krzywej zdefiniowanej na liczbach rzeczywistych.
Krzywa taka składa się z dyskretnych punktów, których współrzędne należą do ciała
na którym jest opisana.
Operacje na krzywej nad ciałem skończonym są zdefiniowane
za pomocą tych samych wzorów algebraicznych, co w przypadku ciała liczb rzeczywistych,
jednak wszystkie działania są wykonywane na ciele $F_p$.
\begin{figure}[!h]
    \centering \includegraphics[width=0.8\linewidth]{sage/ec_2_11-9.png}
    \caption{Krzywa eliptyczna $y^2=x^3-4+2$ nad $GF(2^{11} - 9)$}
\end{figure}


\subsubsection{Problem logarytmu dyskretnego}
Problem logarytmu dyskretnego (\textbf{DLP}) jest
podstawą kryptosystemów oparych o grupy.
Jednymi z bardziej znanych są kryptosystem ElGamala oraz protokół wymiany
kluczy Diffie-Hellmana'a.
\\ Problem logarytmu dyskretnego można zdefiniować na grupach cyklicznych.
zarówno na grupie multiplikatywnej $(\mathbb{G},\cdot)$
oraz grupie addytywnej $(\mathbb{G}, +)$, przy odpowiednim zdefiniowaniu działań grupowych.

Jeżeli G jest grupą cykliczną a $\gamma$ jej generatorem, to logarytmem dyskretnym
elementu $\alpha \in G$ nazywamy najmniejszą nieujemną liczbę całkowitą $x$ taką, że:
\[x = \log_{\gamma}{\alpha}\]

Uważa się, że problem logarytmu dyskretnego jest trudny, ponieważ nie istnieje
algorytm, który znajduje $x$ w czasie wielomianowym\cite{Stinson2021}.


\subsubsection{Problem logarytmu dyskretnego na krzywej eliptycznej}
W przypadku krypografii opartej o krzywe eliptyczne, DLP dotyczy cykliczej
grupy addytywnej $(\mathbb{E},+)$ zdefiniowanej na krzywej eliptycznej.
Aby utworzyć taką grupę, wybieramy punkt $P$ na krzywej eliptycznej $\mathbb{E}$,
który będzie generatorem grupy. Wtedy grupa addytywna $\mathbb{E}$ jest generowana przez
kolejne \textit{potęgi} punktu $P$:
\[\ \langle P \rangle = \{P, 2P, 3P, \ldots, nP = \mathcal{O}\}\]
W takim przypadku, ponieważ operacją na grupie jest dodawanie modulo n, to działanie
potęgowania przedstawia się jako zwielokrotnienie punktu $P$:
\[x \cdot P = Q \textrm{ (mod } n)\]
Analogicznie do problemu logarytmu dyskretnego na grupach multiplikatywnych,
problem logarytmu dyskretnego na krzywej eliptycznej polega na znalezieniu
$x$.
\par
Przy odpowiednim wyborze grupy addytywnej, rozwiązanie problemu logarytmu dyskretnego,
tj. znalezienie $x$,
jest trudne \cite{Stinson2021}.


\subsection{Algorytm Rho Pollarda}
 
RYSUNEK LITERKI RHO (bo jak inaczej bez tego mówic o rho pollardzie :))

DODAC OPIS, co się dzieje po znalezieniu kolizji, jak to pozwala obliczyć logarytm dyskretny

CZY TRZEBA opisywać algorym sekwencyjny skoro i tak korzystam z równoległego? Moze samo odesłanie do literatury
\par
DO UZUPEŁNIENIA
\par
\par
Najszybszym znanym algorytmem rozwiązującym problem logarytmu dyskretnego na krzywej eliptycznej
jest algorytm Rho Pollarda,
zaproponowany przez Johna Pollarda w 1978 roku \cite{Pollard1978}.
Pozwala on na znalezienie logarytmu dyskretnego w czasie $O(\sqrt{n})$,
jednak jest to jedynie czas \textit{oczekiwany}, ponieważ ze względu na losową naturę algorytmu \cite{Blake2005}.
W porównaniu do innego znanego algorytmu, Baby-Step Giant-Step \cite{Stinson2021}, algorytm Rho Pollarda jest bardziej
efektywny pamięciowo, nie wymagając
przestrzeni $O(\sqrt{n})$ a jedynie $O(1)$ w wersji sekwencyjnej \cite{Stinson2021}\cite{Blake2005}.
\par
Klasyczny algorytm Rho Pollarda, oparty o poszukiwanie cyklu, słabo się skaluje w przypadku zrównoleglenia,
osiągając jedynie przyśpieszenie rzędu $O(\sqrt{m})$ dla $m$ procesorów \cite{Goldberg}.
Dlatego w swojej pracy wykorzystałem równoległą wersję algorytmu, zaproponowaną przez Van Oorschota i Wienera \cite{Oorschot}.
% \par
% Idea klasycznego algorytmu Rho Pollarda, polega na poszukiwaniu kolizji punktów:
% $$
% X_i = X_{2i}
% $$
% takich, że:
% $$
% X_i = a_i \cdot P + b_i \cdot Q
% $$
% $$
% X_{2i} = a_{2i} \cdot P + b_{2i} \cdot Q
% $$
% Jeżeli znajdziemy kolizję punktów $X_i$ i $X_j$,
% to za pomocą odpowiednich przekształceń możemy obliczyć logarytm dyskretny $x$:
% $$
% a_i \cdot P + b_i \cdot Q = a_j \cdot P + b_j \cdot Q
% $$
% $$
% (a_i - a_j) \cdot P = (b_j - b_i) \cdot Q
% $$
% $$
% x = \frac{b_j - b_i}{a_i - a_j}
% $$

% Zakładając, że $P$ jest generatorem grupy na krzywej eliptycznej, oraz, że chcemy obliczyć logaryytm dyskretny $x$:
% $$
% x\cdot P = Q
% $$
% Idea algorytmu polega na iteracyjnym stosowaniu losowo wyglądającej funkcji $f$, która generuje kolejne trójki
% w postaci $(X_i, a_i, b_i)$, gdzie $X_i$ jest punktem na krzywej eliptycznej, a $a_i$ oraz $b_i$ są liczbami całkowitymi:
% \[
%     f(X, a, b) =
%     \begin{cases}
%         (X + Q,a, b + 1)               & \text{jeśli } X \in S_1, \\
%         (2X, 2a, 2b) & \text{jeśli } X \in S_2, \\
%         (X + P, a+1, b)                           & \text{jeśli } X \in S_3,
%     \end{cases}
% \]

\subsubsection{Równoległy algorytm Rho Pollarda}
Równoległa wersja algorytmu Rho Pollarda, zakłada zastosowanie wielu równoległych ścieżek błądzenia.
\par
Jednostki obliczeniowe poszukują wtedy specalnych punktów nazywanych wyróźnionymi,
które następnie są przekazywane do centralnego serwera w celu znalezienia kolizji między nimi.
\par
Ponieważ sprawdzenie, czy punkt jest wyróźniony następuje w każdej iteracji,
ważne jest aby kryterium decydujące o uznaniu punktu za wyróźniony, było możliwie
szybkie do sprawdzenia.
Często stosowaną metodą, jest sprawdzanie ilości zer na początku lub na końcu binarnej reprezentacji współrzędnej $x$.
\par
Adding walk zakłada jedynie wykonywanie kolejnych operacji dodawania punktów.

Niech $W_0$ będzie punktem startowym, a $f$ funkcją haszującą $f: \langle P \rangle \rightarrow \{1,..s\}$ o możliwie jednostajnym rozkładzie.
Następnie, potrzebujemy tablicy wstępnie obliczonych punktów:
$R_i = c_i P + d_i Q$ dla $0 \leq i \leq s - 1$.
Funkcja iteracyjna jest wtedy zdefiniowana w następujący sposób:
$$
    W_{i+1} = W_i + R_{f(W_i)}
$$
Ważną kwestią jest też rozmiar tablicy z punktami wstepnie obliczonymi.
Zbyt mały rozmiar powoduje, że funkcja nie będzie dostatecznie losowa.
W eksperymentach praktycznych pokazano, że dla $s \geq 16$, funkcja zapewnia
wystarczający poziom losowości,  niezależnie od rozmiaru grupy.
 \cite{Teske2000}
\par
Istotną zaletą tej funkcji w przypadku programu uruchamianego na GPU,
jest uniknięcie rozgałęzień podczas każdej iteracji, co jest szczególnie istotne w przypadku architektury SIMD.
Prawie zawsze wykonujemy tą samą operację dodawania dwóch punktów,
poza mało prawdopodobnym przypadkiem w którym $W_i == R_{f(W_i)}$.
\par
\newpage
% \newpage % Zaleca się otwieranie rozdziału od nowej strony.
\section{State~of~Art}
\label{sc:state}

\subsection{GPU}
Procesory graficzne są dedykowane do wykonywania wielu równoległych obliczeń.
Dzięki temu, są bardzo wydajne w zadaniach które można łatwo zrónoleglić.
Wiele algorytmów do kryptoanalizy pozwala na przetwarzanie równoległe, 
w szczególności algorytm \textbf{rho-Pollarda}.
\subsubsection*{Solving Discrete Logarithms in Smooth-Order Groups with CUDA}
W roku 2012 na karcie graficznej NVIDIA Tesla M2050 osiągnięto wydajność na poziomie
51.9 miliona operacji mnożenia modularnego 768-bit na sekundę.
Implementacja opierała się głównie na języku C z CUDA framework wraz z jednostkowymi segmentami
w języku PTX który jest zbiorem instrukcji dla CUDA GPU.
Praca ma dla mnie szczególną na tym etapie, ponieważ razem z pracą udostępniono kod
implementacji na prawach open-source, dodatkowo opisuje
ograniczenia i założenia jakie należy uwzględnić przy implementacji
algorytmu rho-Pollarda na GPU\cite{henry-goldberg-cuda}.

\subsubsection*{ECC2K-130 on NVIDIA GPUs}
Artykuł opisuje implementację algorytmu rho-Pollarda na karcie graficznej NVIDIA 
GTX 295.
Autorzy wybrali krzywą Koblitza ECC2K-130.
Opisano decyzje związane z wyborem bazy ( w tym przypadku wybrano bazę normalną).
Przedstawiono również szczegóły związane z zarządzaniem pamięcią oraz problem związany
z DRAM'em karty (przy pełnej utylizacji GPU w pamięci brakowało miejsca na input)
Wynik: Średnio obliczenie ECDLP na tej krzywej zajełoby 2 lata przy 534 kartach.

\subsection{FPGA}
\subsubsection*{Solving Discrete Logarithms in Smooth-Order Groups with CUDA}
W 2014 opublikowano pracę przedstawiającą implementację FPGA
na platformie Virtex-6.
dedykowaną do rozwiązania logarytmu dyskretnego na 113-bitowej krzywej Koblitza.
Opisano zastosowane zabiegi poprawiające optymalizację, oraz design
poszczególnych moduł
Na przykład w celu lepszej optymalizacji, wykorzystano bazę normalną
$F_{2^m}$ w jednym z modułów do liczenia automorfizmu punktów.
Wynik po ekstrapolacji to 28 dni na rozwiązanie logarytmu na krzywej Koblitza 113 bit.

\subsection{CPU}
CPU nie są najwydajniejszą architekturą do wykonywania równoległych obliczeń.
Zazwyczaj charakteryzują się znacznie wydajniejszymi jednostkami obliczeniowymi (rdzeniami)
niż na przykład GPGPU, ale jest ich również znacznie mnniej niż w GPGPU.
CPUs są najlepiej przystosowane do przetwarzania potokowego.
\subsubsection*{A Review on solving ECDLP over Large Finite Field using Parallel
	Pollard's Rho (p) Method}
Praca przedstawia wyniki czasowe przy obliczaniu ECDLP na
ciele skończonym rzędu p do 85-bitów.
Zastosowano do tego cluster CPU o 256 rdzeniach octa-core.
Artykuł również jest interesujący ponieważ zwięźle opsuje background matematyczny
oraz przejrzyście przedstawia wersję równoległą
algorytmu rho Pollarda\cite{rewiev-elliptic-cpu}.
Wynik to 52 godziny dla krzywej na ciele rozmiaru p = 85-bitów.



\newpage
\section{Implementacja}

\subsection*{Ogólna architektura}

Równoległy algorytm Rho Pollarda zakłada architekturę, w której jeden serwer zbiera wygenerowane przez klientów punkty charakterystyczne i porównuje je
w celu znalezienia kolizji. W mojej pracy rolę serwera pełni program działający na CPU, delegujący obliczenia koprocesora, jakim jest GPU.
Klientami generującym punkty charakerystyczne są instancje funkcji iteracyjnej, uruchamiane na GPU, obliczające kolejne punkty podczas spaceru losowego po krzywej eliptycznej.
W celu uproszczenia implementacji kodu po stronie GPU, oraz utyliacji zasobów serwera w czasie oczekiwania na kolejne serie punktów, punkty startowe
stanowiące punkt wejściowy każdej instancji, generowane są po stronie serwera. Za uruchamianie kolejnych serii obliczeń na koprocesorze, odpowiada program uruchomiony w osobnym wątku
komunikujący się z serwerem za pomocą synchronizowanych kolejek.
Taka implementacja architektury, w której asynchronicznie programy nadzorują pracę GPU, pozwala na bardzo proste rozszerzenie obliczeń o kolejne koprocesory.
Wystarczy uruchomić kolejny wątek z \textit{worker'em} wskazując odpowiedni identyfikator GPU.


\subsection*{Artmetyka na ciele $F_{79}$}

\subsubsection{Biblioteka CGBN}

\subsubsection{Reprezentacja punktów 79 bit na platformie CUDA}
Największym słowem bitowych dostępnym natywnie na platformie CUDA, jest 64 bitowy typ danych. Taka reprezentacja nie wystarcza,
do przeprowadzenia operacji na krzywej ECCp79bit. W tym celu należy przedstawić liczbę większą od natywnego rozmiaru, jako wektor wielu słów bitowych
rozmiaru o jeden mniejszego od największego dostępnego.  Przykładowo, punkt na krzywej eliptycznej o współrzędnych rozmiaru 96 bit, można przedstawić w następujący sposób:

\begin{lstlisting}[language=C++]
typedef struct
{
    uint32_t x[3];
    uint32_t y[3];
} EC_point;
\end{lstlisting}

\subsubsection{Dodawanie}

\subsubsection{Mnożenie}

\subsubsection{Odwrotność modulo p}

\subsection*{Funkcja iterująca}

\subsubsection{Wybór sposobu generowanie kolejnych punktów}

\subsubsection{Wstępnie obliczone punkty}
cytowanie about rho pollard walks

% \subsubsection{Reducja Berreta}

\subsubsection{Obliczanie odwrotnosci w seriach}

\subsection{Tailing effect}
\subsubsection{Rozmiar bloku}
\subsubsection{Uruchomienia asynchroniczne}

\subsection{Serwer}

\subsubsection{GPUWorker}

\subsubsection{Komunikacja}

\section{Wyniki}

\subsection{Dalsze usprawnienia}
Redukcja berreta

\subsection{Porównanie z innymi pracami}

%--------------------------------------------
% Literatura
%--------------------------------------------
\newpage
\printbibliography
%--------------------------------------------
% Spisy (opcjonalne)
%--------------------------------------------
\newpage

% Wykaz symboli i skrótów.
% Pamiętaj, żeby posortować symbole alfabetycznie
% we własnym zakresie. Ponieważ mało kto używa takiego wykazu, 
% uznałem, że robienie automatycznie sortowanej listy
% na poziomie LaTeXa to za duży overkill. 
% Makro \acronymlist generuje właściwy tytuł sekcji, 
% w zależności od języka.
% Makro \acronym dodaje skrót/symbol to listy, 
% zapewniając podstawowe formatowanie.
% //AB
\vspace{0.8cm}
\acronymlist
\acronym{EiTI}{Wydział Elektroniki i Technik Informacyjnych}
\acronym{PW}{Politechnika Warszawska}
\acronym{FPGA}{Field Programmable Gates Array}
\acronym{DLP}{Discrete Logarithm Problem}
\acronym{GF}{Galois Field (ciało skończone)}

\listoffigures              % Spis obrazków. 
\vspace{1cm}                % vertical space
\listoftables               % Spis tabel. 
\vspace{1cm}               % vertical space
\lstlistoflistings 		% Spis wydruków
\vspace{1cm}                % vertical space
\listofappendices           % Spis załączników

% Załączniki

%\newpage
%\appendix{Nazwa załącznika 1}
%\lipsum[1]
%
%\newpage
%\appendix{Nazwa załącznika 2}
%\lipsum[1]
% Załączniki

%\newpage

% jesli sa w~Zalacznikach tabele, rysunki, tak na szybko.. i~wylaczyc \listofappendices
%\section*{Załącznik I. Wykaz komend AT czujnika parkowania AN-101D firmy Shenzhen Winext Technology}
%%\appendix{Wykaz komend AT czujnika parkowania AN-101D firmy Shenzhen Winext Technology.}
%\setcounter{section}{1}
%\renewcommand\thetable{I.\arabic{table}}
%\input{tex/zal-1-komendy-at}
%
%\newpage
%\section*{Załącznik II. Ramki komunikacyjne czujnika parkowania AN-101D firmy Shenzhen Winext Technology}
%%\appendix{Ramki komunikacyjne czujnika parkowania AN-101D firmy Shenzhen Winext Technology.}
%\renewcommand\thefigure{II.\arabic{figure}}
%\input{tex/zal-2-ramki-komunikacyjne}




\end{document} % Dobranoc. 

