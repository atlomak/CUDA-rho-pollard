\newpage % Zaleca się otwieranie rozdziału od nowej strony.
\section{Code listings}
\subsection*{Dodawnie punktów na krzywej eliptycznej nad $\mathbb{R}$}

\begin{lstlisting}[language=python, label=sage_1, caption=Dodawnie punktów na krzywej eliptycznej na liczbach rzeczywistych]
# SageMath kernel

sage: R = P + Q
sage: R_inv = -R

# Plot elliptic curve
sage: plot_curve = plot(E, rgbcolor="black", xmax=5)

# Plot points
sage: plot_p = plot(P, marker="o", rgbcolor="red", size=50)
sage: p_label = text(
    "P",
    P.dehomogenize(2),
    horizontal_alignment="right",
    vertical_alignment="bottom",
    color="black",
)
sage: plot_q = plot(Q, marker="o", rgbcolor="red", size=50)
sage: q_label = text(
    " Q",
    Q.dehomogenize(2),
    horizontal_alignment="left",
    vertical_alignment="top",
    color="black",
)
sage: plot_r = plot(R, marker="o", rgbcolor="red", size=50)
sage: r_label = text(
    "R",
    R.dehomogenize(2),
    horizontal_alignment="right",
    vertical_alignment="bottom",
    color="black",
)
sage: plot_r_inv = plot(R_inv, marker="o", rgbcolor="red", size=50)
sage: r_inv_label = text(
    "-R",
    R_inv.dehomogenize(2),
    horizontal_alignment="right",
    vertical_alignment="top",
    color="black",
)

sage: p6 = line2d(
    [P.dehomogenize(2), R_inv.dehomogenize(2)], linestyle="--", rgbcolor="blue"
)
sage: p7 = line2d(
    [R.dehomogenize(2), R_inv.dehomogenize(2)], linestyle="--", rgbcolor="blue"
)

sage: plot_curve + plot_p + plot_q + plot_r + plot_r_inv + p6 + p7 + p_label + q_label + r_label + r_inv_label
\end{lstlisting}

\subsection*{Dodawnie punktów na krzywej eliptycznej nad $\mathbb{GF}(7)$}
\begin{lstlisting}[language=python, label=sage_2, caption=Dodawnie punktów na krzywej eliptycznej zdefiniowanej na GF(7)]
Gf = GF(7)

E = EllipticCurve(Gf,[3,4])

def ell_add(E, P1, P2):
    a, b, p = E
    if P1 == "inf": return P2
    if P2 == "inf": return P1
    x1, y1 = P1; x2,y2 = P2
    x1 %= p; y1 %= p; x2 %= p; y2 %= p;

    if x1 == x2 and y1 == p-y2: return "inf"

    if P1 == P2:
        if y1 == 0: return "inf"
        lam = (3*x1^2+a) * inverse_mod(2*y1,p)
    else:
        lam = (y1-y2) * inverse_mod(x1-x2,p)
    x3 = lam^2 - x1 - x2
    y3 = -lam*x3 - y1 + lam*x1
    return (x3%p, y3%p)

P1 = E.random_point()

p1 = (2,5)
print(p1)
for _ in range(9):
    p1 = ell_add(e, p1, p1)
    p1 = (Integer(p1[0]),Integer(p1[1]))
    print(p1)


\end{lstlisting}

\subsection*{Kryptosystem El-Gamala na $(\mathbb{G}, \cdot)$}
\begin{lstlisting}[language=python, label=sage_3, caption=Kryptosystem El-Gamala]
G = Integers(11)

# Bob:

# Private
x = G(3)

## Public
g = G(5)
h = g^x

# Alice:

# Message
m = G(5)

# Private
k = G(8)

# Public
a = g^k
b = (h^k)*m

# Decrypt Bob:
b*(a^(-1))^x
\end{lstlisting}

\subsection*{Kryptosystem El-Gamala na krzywej $y^2 = x^3 + x + 6 \in \mathbb{GF}(11)$ }
\begin{lstlisting}[language=python, label=sage_3, caption=Kryptosystem El-Gamala na krzywej]
Gf = GF(11)
E = EllipticCurve(Gf, [1, 6])

# Alice -> Bob

# Bob private
m = 7

# Bob public
P = E(2, 7)
Q = m * P


def alice_encrypt():
    k = 6  # random secret variable
    x = 9  # message

    kP = k * P
    kQ = k * Q

    secret_hash = safe_hash_function(kQ)

    y1 = kP
    y2 = (x + secret_hash) % 11
    return (y1, y2)


# just mock, should be secure one way hash fucntion
def safe_hash_function(q):
    if q == E(8, 3):
        return 4
    return 3


# Alice sends cypher
cypher = alice_encrypt()


def bob_decrypt(cypher):
    y1 = cypher[0]
    y2 = cypher[1]

    kQ = m * y1  # m * (kP) / but mP = Q so its kQ

    secret_hash = safe_hash_function(kQ)

    plain_message = (y2 - secret_hash) % 11

    return plain_message


bob_decrypt(cypher)
\end{lstlisting}



\subsection*{Algorytm rho-Pollarda dla grupy multiplikatywnej}
\begin{lstlisting}[language=python, label=sage_4, caption=Algorytm rho-Pollarda dla grupy multiplikatywnej]
class Group_parameters:
    def __init__(self, p, n, alpha, beta) -> None:
        self.p = p
        self.n = n
        self.alpha = alpha
        self.beta = beta

class Triple:
    def __init__(self, x, a, b) -> None:
        self.x = x
        self.a = a
        self.b = b
    def __str__(self) -> str:
        return f"x = {self.x}, a = {self.a}, b = {self.b}"


def f(tripe: Triple, group: Group_parameters) -> Triple:
    if tripe.x % 3 == 1:
        x = group.beta * tripe.x % group.p
        a = tripe.a
        b = tripe.b + 1
        return Triple(x, a, b)
    if tripe.x % 3 == 0:
        x = tripe.x ** 2 % group.p
        a = tripe.a * 2
        b = tripe.b * 2
        return Triple(x, a, b)
    else:
        x = group.alpha * tripe.x % group.p
        a = tripe.a + 1
        b = tripe.b
        return Triple(x, a, b)

g = Group_parameters(809, 101, 89, 618)

t1 = f(Triple(1, 0, 0), g)
t2 = f(t1, g)


i = 1
print('%s %s | %s %s' % (i, t1, 2*i, t2))

i = 2
while(t1.x != t2.x):
    t1 = f(t1, g)
    t2 = f(f(t2, g), g)
    print('%s %s | %s %s' % (i, t1, 2*i, t2))
    i=i+1

print(f"Znaleziona kolizja: x: {t1.x}")
\end{lstlisting}

\subsection*{Algorytm rho-Pollarda znajdowania punktów na krzywej eliptycznej}
\begin{lstlisting}[language=python, label=sage_5, caption=Algorytm rho-Pollarda znajdowania punktów na krzywej eliptycznej]
class Group_parameters:
    def __init__(self, p, alpha, beta) -> None:
        self.p = p
        self.alpha = alpha
        self.beta = beta


class Triple:
    def __init__(self, x, a, b) -> None:
        self.x = x  # Point at Elliptic Curve
        self.a = a  # just a number
        self.b = b  # just a number

    def __str__(self) -> str:
        return f"x = {self.x}, a = {self.a}, b = {self.b}"


def f(tripe: Triple, group: Group_parameters) -> Triple:
    #print("Function run")
    #print(f"Input: {tripe}")

    x_of_xpoint = tripe.x[0]
    y_of_xpoint = tripe.x[1]

    p = group.p

    if int(y_of_xpoint) % 3 == 1:
        x = group.beta + tripe.x
        a = tripe.a
        b = tripe.b + 1
        # check 
        if (x != a * g.alpha + b * g.beta):
            pass
            print(f"1 xab: {x} {a} {b}")
        return Triple(x, a, b)
    if int(y_of_xpoint) % 3 == 0:
        x = 2 * tripe.x
        a = (tripe.a * 2)
        b = (tripe.b * 2)
        if (x != a * g.alpha + b * g.beta):
            pass
            print(f"2 xab: {x} {a} {b}")
        return Triple(x, a, b)
    else:
        x = group.alpha + tripe.x
        a = tripe.a + 1
        b = tripe.b
        if (x != a * g.alpha + b * g.beta):
            pass
            print(f"3 xab: {x} {a} {b}")
        return Triple(x, a, b)


def main(g: Group_parameters, t1):
    i = 1

    t2 = f(t1, g)

    print("%s %s | %s %s" % (i, t1, 2 * i, t2))

    i = 2
    while t1.x != t2.x:
        t1 = f(t1, g)
        t2 = f(f(t2, g), g)
        print("%s %s | %s %s" % (i, t1, 2 * i, t2))
        i = i + 1

    print(f"Found:\nt1: {t1}\nt2: {t2}")

    x = -((t2.a - t1.a) / (t1.b - t2.b))

    print(x)
\end{lstlisting}